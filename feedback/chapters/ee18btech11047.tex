\begin{enumerate}[label=\thesection.\arabic*.,ref=\thesection.\theenumi]
\numberwithin{equation}{enumi}

\item For the circuit shown in Fig. \ref{fig:ee18btech11047_fig1}, find $L(s)$, $L(j\omega)$, the frequency for zero loop phase, and $R_{2}/R_{1}$ for oscillation.
\renewcommand{\thefigure}{\theenumi.\arabic{figure}}
%
\begin{figure}[!ht]
	\begin{center}
		\resizebox{\columnwidth}{!}{\input{./figs/ee18btech11047/ee18btech11047_1.tex}}
	\end{center}
\caption{}
\label{fig:ee18btech11047_fig1}
\end{figure}
%
\\
\solution The equivalent control system representation is shown in Fig. \ref{fig:ee18btech11047_fig2}. Oscillators do not include input signal.
\begin{figure}[!ht]
	\begin{center}
		\resizebox{\columnwidth}{!}{\input{./figs/ee18btech11047/ee18btech11047_2.tex}}
	\end{center}
\caption{}
\label{fig:ee18btech11047_fig2}
\end{figure}
\renewcommand{\thefigure}{\theenumi}
\item Find the open loop gain $G$.\\
\solution Let the closed loop gain, open-loop gain of op-amp connected in non-inverting configuration be $T_{1}$ and $G_{1}$ respectively.
From Table \ref{table:ee18btech11005_Output_Table}
\begin{align}
T_{1} &= \frac{G_{1}\brak{R_1+R_2}}{\brak{R_1+R_2}+G_{1}R_1}
\end{align}
\begin{align}
T_{1} &= \frac{\brak{R_1+R_2}}{\brak{R_1+R_2}/G_{1}+R_1}
\end{align}
Assuming $G_{1}\to\infty$
\begin{align}
T_{1} &= 1 + \frac{R_{2}}{R_{1}}
\end{align}
The open loop gain of the circuit shown in Fig. \ref{fig:ee18btech11047_fig1} is equal to the closed loop gain of an op-amp connected in non-inverting configuration.
\begin{align}
G &= T_{1}
\end{align}
\begin{align}
\label{eq:ee18btech11047_1}
\implies G = 1 + \frac{R_{2}}{R_{1}}
\end{align}
%
\item Find the feedback factor $H$. \\
\solution The small signal model is shown in Fig. \ref{fig:ee18btech11047_fig3}
\begin{figure}[!ht]
	\begin{center}
		\resizebox{\columnwidth}{!}{\input{./figs/ee18btech11047/ee18btech11047_3.tex}}
	\end{center}
\caption{}
\label{fig:ee18btech11047_fig3}
\end{figure}
Applying KCL at node $V_f$
\begin{align}
\frac{V_{f} - 0}{\frac{1}{sC}} +\frac{V_{f} - V_{a}}{R} &= 0
\end{align}
\begin{align}
V_{f}\brak{sC+\frac{1}{R}} &= \frac{V_{a}}{R} 
\end{align}
\begin{align}
\label{eq:ee18btech11047_2}
V_{a} &= V_{f}\brak{sRC + 1} 
\end{align}
Applying KCL at node $V_{a}$
\begin{align}
\frac{V_{a} - V_{f}}{R} + \frac{V_{a} - 0}{R} + \frac{V_{a} - V_{o}}{\frac{1}{sC}} &= 0
\end{align}
\begin{align}
V_{a}\brak{\frac{2}{R} +sC} &= \frac{V_{f}}{R} + V_{o}sC
\end{align}
Substitute $V_{a}$ value from equation\eqref{eq:ee18btech11047_2}
\begin{align}
V_{f}(sRC + 1)\brak{\frac{2}{R} + sC} &= \frac{V_{f}}{R} + V_{o}sC
\end{align}
\begin{align}
V_{f}\brak{3 + sRC + \frac{1}{sRC}} &= V_{o}
\end{align}
The feedback factor H is given by 
\begin{align}
H &= \frac{V_{f}}{V_{o}}
\end{align}
\begin{align}
\label{eq:ee18btech11047_3}
\implies H &= \frac{1}{\brak{3+sRC +\frac{1}{sRC}}}
\end{align}
%
\item Find the loop gain L(s).\\
\solution The transfer function of the equivalent positive feedback circuit in Fig. \ref{fig:ee18btech11047_fig2} is  
\begin{align}
T &= \frac{G}{1-GH}
\end{align}
Therefore, loop gain is given by 
\begin{align}
L &= GH
\end{align}
From equations \eqref{eq:ee18btech11047_1} and \eqref{eq:ee18btech11047_3}
\begin{align}
L(s) &= \brak{1 + \frac{R_{2}}{R_{1}}}\brak{\frac{1}{3+sRC+\frac{1}{sRC}}}
\end{align}
\begin{align}
\label{eq:ee18btech11047_4}
\implies L(s) &= \brak{\frac{1+\frac{R_{2}}{R_{1}}}{3+sRC+\frac{1}{sRC}}}
\end{align}
%
\item Find the loop gain in terms of $j\omega$ .\\
\solution Substitute $s = j\omega$ in equation \eqref{eq:ee18btech11047_4}
\begin{align} 
L(j\omega)&= \brak{\frac{1+\frac{R_{2}}{R_{1}}}{3+j\omega RC+\frac{1}{j\omega RC}}}
\end{align}
\begin{align}
\label{eq:ee18btech11047_5}
\implies L(j\omega)&= \brak{\frac{1+\frac{R_{2}}{R_{1}}}{3+ j \brak{\omega RC-\frac{1}{\omega RC}}}}
\end{align}
\item Find the frequency for zero loop phase.\\
\solution The frequency at which loop phase will be zero (i.e. loop gain will be a real number).To obtain the required frequency, equate the imaginary part of the loop gain $L(j \omega )$ to zero.
\begin{align}
j\brak{\omega RC - \frac{1}{\omega RC}} &= 0
\end{align}
\begin{align}
\omega^{2} &= \frac{1}{(RC)^{2}}
\end{align}
\begin{align}
\implies \omega &= \frac{1}{RC}
\end{align}
%
\item Find $R_{2}/R_{1}$ for oscillation.\\
\solution For oscillations to start, 
\begin{itemize}
    \item the imaginary part of the loop gain should become zero.
    \item the loop gain must be made greater than unity.
\end{itemize}
From equation \eqref{eq:ee18btech11047_5} 
\begin{align}
\brak{\frac{1+\frac{R_{2}}{R_{1}}}{3+j(0)}} &> 1
\end{align}
\begin{align}
1+\frac{R_{2}}{R_{1}} &> 3
\end{align}
\begin{align}
\implies \frac{R_{2}}{R_{1}} &> 2
\end{align}
\end{enumerate}
