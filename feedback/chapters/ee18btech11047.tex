\begin{enumerate}[label=\thesection.\arabic*.,ref=\thesection.\theenumi]
\numberwithin{equation}{enumi}

\item For the circuit shown in Fig. \ref{fig:ee18btech11047_fig1}, find $L(s)$, $L(j\omega)$, the frequency for zero loop phase, and $R_{2}/R_{1}$ for oscillation.
\renewcommand{\thefigure}{\theenumi.\arabic{figure}}
%
\begin{figure}[!ht]
	\begin{center}
		\resizebox{\columnwidth}{!}{\begin{circuitikz}
\ctikzset{bipoles/length=1cm}

 
\draw (0, 0) node[op amp] (opamp) {};
\draw (opamp.-) --(-1,0.35)-- (-1,1) to[R=$R_2$,*-*] (1,1) -- (1,0) -- (1,-1) to [C=$C$,*-*] (0.25,-1) to [R=$R$,*-*] (-1,-1) -- (-1,-0.35) to (opamp.+);
\draw (-1,1) to[R=$R_1$,*-*] (-3.5, 1) to node[ground]{}  (-3.5, 0.9) ;
\draw (0.25,-1) to [R = $R$,*-*] (0.25,-3) to node[ground]{} (0.25,-3);
\draw (-1,-1) to[C=$C$,*-*] (-1, -3) to node[ground]{} (-1,-3);
\draw (opamp.out) -- (1,0) ;
%node at (1.5,0){};
\end{circuitikz}

}
	\end{center}
\caption{}
\label{fig:ee18btech11047_fig1}
\end{figure}
%
\\
\solution The equivalent control system representation is shown in Fig. \ref{fig:ee18btech11047_fig2}. Oscillators do not include input signal.
\begin{figure}[!ht]
	\begin{center}
		\resizebox{\columnwidth}{!}{\tikzstyle{block} = [draw, fill=blue!20, rectangle, 
    minimum height=3em, minimum width=6em]
\tikzstyle{sum} = [draw, fill=blue!20, circle, node distance=1cm]
\tikzstyle{input} = [coordinate]
\tikzstyle{output} = [coordinate]
\tikzstyle{pinstyle} = [pin edge={to-,thin,black}]

\begin{tikzpicture}[auto, node distance=2cm,>=latex']
    \node [input, name=input] {};
    \node [sum, right of=input] (sum) {};
    \node [block, right of=sum] (controller) {$G$};
    \node [output, right of=controller] (output) {};
    \node [block, below of=controller] (feedback) {$H$};
    \draw [draw,->] (input) -- node {} (sum);
    \draw [->] (sum) -- node {$V_i$} (controller);
    \draw [->] (controller) -- node [name=y] {$V_o$}(output);
    \draw [->] (y) |- (feedback);
    \draw [->] (feedback) -| node[pos=0.99]{$+$}  node [near end] {$V_f$} (sum);
\end{tikzpicture}
}
	\end{center}
\caption{}
\label{fig:ee18btech11047_fig2}
\end{figure}
\renewcommand{\thefigure}{\theenumi}
\item Find the open loop gain $G$.\\
\solution Let the closed loop gain, open-loop gain of op-amp connected in non-inverting configuration be $T_{1}$ and $G_{1}$ respectively.
From Table \ref{table:ee18btech11005_Output_Table}
\begin{align}
T_{1} &= \frac{G_{1}\brak{R_1+R_2}}{\brak{R_1+R_2}+G_{1}R_1}
\end{align}
\begin{align}
T_{1} &= \frac{\brak{R_1+R_2}}{\brak{R_1+R_2}/G_{1}+R_1}
\end{align}
Assuming $G_{1}\to\infty$
\begin{align}
T_{1} &= 1 + \frac{R_{2}}{R_{1}}
\end{align}
The open loop gain of the circuit shown in Fig. \ref{fig:ee18btech11047_fig1} is equal to the closed loop gain of an op-amp connected in non-inverting configuration.
\begin{align}
G &= T_{1}
\end{align}
\begin{align}
\label{eq:ee18btech11047_1}
\implies G = 1 + \frac{R_{2}}{R_{1}}
\end{align}
%
\item Find the feedback factor $H$. \\
\solution The small signal model is shown in Fig. \ref{fig:ee18btech11047_fig3}
\begin{figure}[!ht]
	\begin{center}
		\resizebox{\columnwidth}{!}{\usetikzlibrary{decorations.markings}
\begin{circuitikz}
\ctikzset{bipoles/length=1cm}

\draw 
(1.5,1) to [C=$C$] (1.5,5) to (1.5,5)  node[ground,rotate=180]{} 
(1.5,2) to [R=$R$] (3.5,2) to [R=$R$,*-*] (3.5,4) to (3.5,5) node[ground,rotate=180]{} 
(3.5,2) to [C=$C$] (5,2) -- (5,1)
%(1.5,3) node[pos=10]{$V_i$}
(1.5,-1.25)  node at(1.7,-1.25){$-$} 
(1.5,-1.25) -- (1,-1.25) -- (1,-1.75) to[R=$R_1$] (1,-2.75) --(1,-3) node[ground]{}
(1,-1.5) to[R=$R_2$,*-*] (5,-1.5) {}
(5,-1.5) -- (5,1) --(3.5,1) to[V=$G_{0}V_i$] (3.5,-0.5) node[ground]{}
(5,1) --(6,1)
(6,1) --(6.5,1) node at(6.8,1){$V_o$}
node at (1.8,-0.3) {$V_i$}
node at (3.5,1.7) {$V_{a}$}
node at (1.1,2) {$V_{f}$}
node at (0.65,-1.5){$V_{f2}$}
node at(1.8,1){$+$}
;\end{circuitikz}
}
	\end{center}
\caption{}
\label{fig:ee18btech11047_fig3}
\end{figure}
Applying KCL at node $V_f$
\begin{align}
\frac{V_{f} - 0}{\frac{1}{sC}} +\frac{V_{f} - V_{a}}{R} &= 0
\end{align}
\begin{align}
V_{f}\brak{sC+\frac{1}{R}} &= \frac{V_{a}}{R} 
\end{align}
\begin{align}
\label{eq:ee18btech11047_2}
V_{a} &= V_{f}\brak{sRC + 1} 
\end{align}
Applying KCL at node $V_{a}$
\begin{align}
\frac{V_{a} - V_{f}}{R} + \frac{V_{a} - 0}{R} + \frac{V_{a} - V_{o}}{\frac{1}{sC}} &= 0
\end{align}
\begin{align}
V_{a}\brak{\frac{2}{R} +sC} &= \frac{V_{f}}{R} + V_{o}sC
\end{align}
Substitute $V_{a}$ value from equation\eqref{eq:ee18btech11047_2}
\begin{align}
V_{f}(sRC + 1)\brak{\frac{2}{R} + sC} &= \frac{V_{f}}{R} + V_{o}sC
\end{align}
\begin{align}
V_{f}\brak{3 + sRC + \frac{1}{sRC}} &= V_{o}
\end{align}
The feedback factor H is given by 
\begin{align}
H &= \frac{V_{f}}{V_{o}}
\end{align}
\begin{align}
\label{eq:ee18btech11047_3}
\implies H &= \frac{1}{\brak{3+sRC +\frac{1}{sRC}}}
\end{align}
%
\item Find the loop gain L(s).\\
\solution The transfer function of the equivalent positive feedback circuit in Fig. \ref{fig:ee18btech11047_fig2} is  
\begin{align}
T &= \frac{G}{1-GH}
\end{align}
Therefore, loop gain is given by 
\begin{align}
L &= GH
\end{align}
From equations \eqref{eq:ee18btech11047_1} and \eqref{eq:ee18btech11047_3}
\begin{align}
L(s) &= \brak{1 + \frac{R_{2}}{R_{1}}}\brak{\frac{1}{3+sRC+\frac{1}{sRC}}}
\end{align}
\begin{align}
\label{eq:ee18btech11047_4}
\implies L(s) &= \brak{\frac{1+\frac{R_{2}}{R_{1}}}{3+sRC+\frac{1}{sRC}}}
\end{align}
%
\item Find the loop gain in terms of $j\omega$ .\\
\solution Substitute $s = j\omega$ in equation \eqref{eq:ee18btech11047_4}
\begin{align} 
L(j\omega)&= \brak{\frac{1+\frac{R_{2}}{R_{1}}}{3+j\omega RC+\frac{1}{j\omega RC}}}
\end{align}
\begin{align}
\label{eq:ee18btech11047_5}
\implies L(j\omega)&= \brak{\frac{1+\frac{R_{2}}{R_{1}}}{3+ j \brak{\omega RC-\frac{1}{\omega RC}}}}
\end{align}
\item Find the frequency for zero loop phase.\\
\solution The frequency at which loop phase will be zero (i.e. loop gain will be a real number).To obtain the required frequency, equate the imaginary part of the loop gain $L(j \omega )$ to zero.
\begin{align}
j\brak{\omega RC - \frac{1}{\omega RC}} &= 0
\end{align}
\begin{align}
\omega^{2} &= \frac{1}{(RC)^{2}}
\end{align}
\begin{align}
\implies \omega &= \frac{1}{RC}
\end{align}
%
\item Find $R_{2}/R_{1}$ for oscillation.\\
\solution For oscillations to start, 
\begin{itemize}
    \item the imaginary part of the loop gain should become zero.
    \item the loop gain must be made greater than unity.
\end{itemize}
From equation \eqref{eq:ee18btech11047_5} 
\begin{align}
\brak{\frac{1+\frac{R_{2}}{R_{1}}}{3+j(0)}} &> 1
\end{align}
\begin{align}
1+\frac{R_{2}}{R_{1}} &> 3
\end{align}
\begin{align}
\implies \frac{R_{2}}{R_{1}} &> 2
\end{align}
\end{enumerate}
