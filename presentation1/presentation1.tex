\documentclass[15pt]{beamer}

%~~~~~~~~~~~~~~~~~~~~~~~~~~~~~~~~~~~~~~~~~~~~~~~~~~~~~~~~~~~~~~~~~~~~~~~~~~~~~~
% Use roboto Font (recommended)
\usepackage[sfdefault]{roboto}
\usepackage[utf8]{inputenc}
\usepackage[T1]{fontenc}
%~~~~~~~~~~~~~~~~~~~~~~~~~~~~~~~~~~~~~~~~~~~~~~~~~~~~~~~~~~~~~~~~~~~~~~~~~~~~~~

%~~~~~~~~~~~~~~~~~~~~~~~~~~~~~~~~~~~~~~~~~~~~~~~~~~~~~~~~~~~~~~~~~~~~~~~~~~~~~~
% Define where theme files are located. ('/styles')
\usepackage{styles/fluxmacros}
\usefolder{styles}
% Use Flux theme v0.1 beta
% Available style: asphalt, blue, red, green, gray 
\usetheme[style=asphalt]{flux}
%~~~~~~~~~~~~~~~~~~~~~~~~~~~~~~~~~~~~~~~~~~~~~~~~~~~~~~~~~~~~~~~~~~~~~~~~~~~~~~

%~~~~~~~~~~~~~~~~~~~~~~~~~~~~~~~~~~~~~~~~~~~~~~~~~~~~~~~~~~~~~~~~~~~~~~~~~~~~~~
% Extra packages for the demo:
\usepackage{booktabs}
\usepackage{colortbl}
\usepackage{ragged2e}
\usepackage{schemabloc}
%~~~~~~~~~~~~~~~~~~~~~~~~~~~~~~~~~~~~~~~~~~~~~~~~~~~~~~~~~~~~~~~~~~~~~~~~~~~~~~

%~~~~~~~~~~~~~~~~~~~~~~~~~~~~~~~~~~~~~~~~~~~~~~~~~~~~~~~~~~~~~~~~~~~~~~~~~~~~~~
% Informations
\title{\hspace{2pt} EE2227-CONTROL SYSTEMS}
%\subtitle{Modern theme v0.1}
\author{Tejaswini}
\institute{EE18BTECH11047}
\date{\today}
\titlegraphic{assets/overleaf.png}
%~~~~~~~~~~~~~~~~~~~~~~~~~~~~~~~~~~~~~~~~~~~~~~~~~~~~~~~~~~~~~~~~~~~~~~~~~~~~~~

\begin{document}

% Generate title page
\titlepage


\begin{frame}{QUESTION}{EE 2017(SET-2)}
	\justifying  \textbf{QUESTION-41} \newline Consider the system described by the following state space representation 
	\newline \newline \begin{bmatrix}\dot x_{1}(t)\\\dot x_{2}(t)
	\end{bmatrix} = \begin{bmatrix}0 & 1 \\ 0 & -2\end{bmatrix}\begin{bmatrix}x_{1}(t)\\x_{2}(t)\end{bmatrix} + \begin{bmatrix}0 \\ 1\end{bmatrix}u(t) \newline \newline \newline
	y(t) = \begin{bmatrix}1 & 0\end{bmatrix}\begin{bmatrix}x_{1}(t)\\x_{2}(t)\end{bmatrix}
	\newline \newline If u(t) is a unit step input and \begin{bmatrix}x_{1}(0) \\ x_{2}(0)\end{bmatrix} = \begin{bmatrix}1 \\ 0 \end{bmatrix} , the value of output y(t) at t = 1 sec(rounded off to three decimal places) is __________
\end{frame}

\def\beamer@mytheme@style{green}
\begin{frame}[fragile]{SOLUTION}
	From the given ,\newline 
	\newline Let\hspace{10pt}A = \begin{bmatrix}0 & 1\\0 & -2\end{bmatrix}\hspace{5pt}B = \begin{bmatrix}0\\1\end{bmatrix}\hspace{5pt}C=\begin{bmatrix}1 & 0 \end{bmatrix}
	\newline \newline \newline X(t) = \begin{bmatrix}x_{1}(t)\\x_{2}(t)\end{bmatrix}
	\newline \newline \newline $\dot X(t)$ = \begin{bmatrix}\dot x_{1}(t) \\ \dot x_{2}(t)\end{bmatrix}\hspace{15pt} where $\dot x(t)$ = $\frac{d}{dt}$(x(t))
	
	
	
\end{frame}

\subsection{fonts}



\subsection{footnotes}
\begin{frame}{SOLUTION}\textbf{LAPLACE TRANSFORMS}
\newline \newline L\{$\dot f(t)$\} = sF(s) - f(0)
\newline \newline L\{u(t)\} = $\frac{1}{s}$
\newline \newline L\{t\} = $\frac{1}{s^2}$
\newline \newline L\{e^{at}\} = \frac{1}{s-a}
\newline \newline L\{e^{-at}\} = \frac{1}{s+a}
\end{frame}
\begin{frame}{SOLUTION}
		$\dot X(t)$ = Ax(t) + Bu(t)\hspace{70pt}------(1)
		\newline \newline Laplace transform on equation (1) results in 
		\newline \newline sX(s) - x(0) \hspace{5pt}= \hspace{5pt}AX(s) + Bu(s)
		\newline \newline (sI - A)X(s) \hspace{5pt}= \hspace{5pt}x(0) +B$\frac{1}{s}$
		\newline \newline X(s)\hspace{5pt} = \hspace{5pt}(sI - A)^{-1}[x(0) + B\frac{1}{s}]\hspace{55pt}--(2)
\end{frame}

\section{Collections}
\subsection{lists}

\begin{frame}{SOLUTION}
   sI - A \hspace{5pt}= \hspace{5pt}s\begin{bmatrix}1 & 0\\0 & 1\end{bmatrix} - \begin{bmatrix}0 & 1 \\ 0 & -2\end{bmatrix} \hspace{5pt}=\hspace{5pt} \begin{bmatrix}s & -1 \\ 0 & s+2\end{bmatrix}
   \newline \newline (sI - A)^{-1} \hspace{5pt}=\hspace{5pt} \frac{1}{s(s+2)}\begin{bmatrix}s+2 & 1 \\ 0 & s\end{bmatrix}\hspace{5pt}=\hspace{5pt}\begin{bmatrix}\frac{1}{s} & \frac{1}{s(s+2)}\\0&\frac{1}{s+2}\end{bmatrix}
   \newline \newline Substituting \hspace{5pt}(sI - A)^{-1}\hspace{5pt} in \hspace{5pt} (2)
   \newline \newline X(s) = \begin{bmatrix}\frac{1}{s} & \frac{1}{s(s+2)} \\ 0 & \frac{1}{s+2}\end{bmatrix}\begin{bmatrix}\begin{bmatrix}1 \\ 0\end{bmatrix} + \frac{1}{s}\begin{bmatrix}0 \\ 1\end{bmatrix}\end{bmatrix} \hspace{15pt} since, \begin{bmatrix}x_{1}(0)\\x_{2}(0)\end{bmatrix} = \begin{bmatrix}1\\0\end{bmatrix}
\end{frame}

\subsection{tables}

\begin{frame}{SOLUTION}
X(s)\hspace{15pt}=\hspace{15pt}\begin{bmatrix}\frac{1}{s} & \frac{1}{s(s+2)} \\ 0 & \frac{1}{s+2}\end{bmatrix}\begin{bmatrix}1 \\ \frac{1}{s}\end{bmatrix}
\newline \newline \newline X(s) \hspace{15pt}= \hspace{15pt}\begin{bmatrix}\frac{1}{s} + \frac{1}{s^{2}(s+2)} \\ \frac{1}{s(s+2)}\end{bmatrix}
\newline \newline On splitting into partial fractions
\newline \newline X(s) \hspace{15pt}= \hspace{15pt}\begin{bmatrix}\frac{1}{4(s+2)}+ \frac{3}{4s} + \frac{1}{2s^{2}} \\ \frac{1}{2s} - \frac{1}{2(s+2)}\end{bmatrix}
\end{frame}

\subsection{blocs}

\begin{frame}[fragile]{SOLUTION}
  	Inverse laplace transform on X(s) .....
  	\newline \newline L^{-1}\{X(s)\} \hspace{10pt}=\hspace{10pt} \begin{bmatrix}L^{-1}\{ \frac{1}{4(s+2)}+ \frac{3}{4s} + \frac{1}{2s^{2}} \} \\ L^{-1}\{\frac{1}{2s} - \frac{1}{2(s+2)} \}\end{bmatrix}
  	\newline \newline \newline x(t) \hspace{25pt}=\hspace{10pt} \begin{bmatrix}\frac{1}{4}e^{-2t} + \frac{3}{4} + \frac{1}{2}t \\ \frac{1}{2} - \frac{1}{2}e^{-2t}\end{bmatrix}
  	
\end{frame}

\begin{frame}[fragile]{SOLUTION}
Given that y(t) = C x(t)
  \newline \newline y(t) = \begin{bmatrix}1 & 0 \end{bmatrix}\begin{bmatrix}\frac{1}{4}e^{-2t} + \frac{3}{4} + \frac{1}{2}t \\ \frac{1}{2} - \frac{1}{2}e^{-2t}\end{bmatrix}
  \newline \newline y(t) = $\frac{1}{4}e^{-2t}$ + $\frac{3}{4}$ + $\frac{1}{2}t$
  \newline \newline y(1) = $\frac{1}{4}e^{-2}$ + $\frac{3}{4}$ + $\frac{1}{2}(1)$
  \newline \newline y(1) = 1.28383
  \newline \newline Rounding off to three decimals......
  \newline \newline y(1) = 1.284
  
\end{frame}

\end{document}
